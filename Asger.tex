\documentclass{article}
\usepackage[utf8]{inputenc}

\usepackage{amsmath}
\usepackage{amssymb}
\usepackage{mathrsfs}
\usepackage{minted}
\usepackage[shortlabels]{enumitem}

% pakker for at lave bokse
\usepackage{blindtext}
\usepackage{tcolorbox}
\usepackage{graphicx}

\usepackage{hyperref}
\hypersetup{
  colorlinks   = true, %Colours links instead of ugly boxes
  urlcolor     = blue, %Colour for external hyperlinks
  linkcolor    = black, %Colour of internal links
  citecolor   = lightgray %Colour of citations
}

\makeatletter
\newcommand\xleftrightarrow[2][]{%
  \ext@arrow 9999{\longleftrightarrowfill@}{#1}{#2}}
\newcommand\longleftrightarrowfill@{%
  \arrowfill@\leftarrow\relbar\rightarrow}
\makeatother

\begin{document}
\input{Titlepage.tex}
\tableofcontents
\section{Introduction}
    \subsection{Leader election algorithms}
        
        \subsubsection{The Bully algorithm}
            In the domain of leader election algorithms, one widely recognized solution is the Bully Algorithm, originally proposed by Garcia-Molina in \cite{molina:1982}. The fundamental concept underlying this algorithm revolves around a set of \(N\) processes denoted as \(\{P_{0},...,P_{N-1}\}\), each possessing a unique identifier (UID), where \(id(P_{k}) =k\). When a process discerns the absence of responses from the current leader, it triggers an election procedure. The process \(P_{k}\) initiates an election in the subsequent manner:
            

            
\section{Methods and materials}
    
\section{Experiments, results, and discussion}

\section{Conclusion and perspectives}

\newpage
    \bibliographystyle{IEEEtran} % We choose the "plain" reference style
    \bibliography{P1} % Entries are in the refs.bib file
\end{document}
